\documentclass[format=sigchi, screen=true, review=true]{acmart}
% rubber: clean TouchLib-paper.brf TouchLib-paper.err TouchLib-paper.out TouchLib-paper.thm TouchLib-paper-blx.bib TouchLib-paper.run.xml TouchLib-paper.xcp
% rubber: depend TouchLib-paper.bib
% rubber: module bibtex
% rubber: module index
\usepackage[english]{babel}
%\usepackage[square,numbers]{natbib}
\usepackage{natbib}

\begin{document}
\bibliography{TouchLib-paper}
\bibliographystyle{abbrvnat}
\acmConference[CHI]{ACM CHI Conference on Human Factors in Computing Systems}{May 04--09, 2019}{Glasgow, UK}
\title{TouchLib: a portable and user-friendly library for robust capacitive sensing}
\author{Admar Schoonen}
\affiliation{Eindhoven University of Technology, department of Industrial Design}
\email{admar@familieschoonen.nl}
\maketitle
\section{Introduction}
Capacitive sensors as part of a human interface device (HID) have been used for
a long time. Their attributes such as low power, low cost, ability to place the
sensor behind smooth and flush surfaces and the fact that no force at all is
required to activate the sensor, giving the user an almost magical experience,
all contribute to their widespread use. FIXME: add references to early
applications, iPod and iPhone?

However, capacitive sensors are also known to be difficult to work with since
the signals can be very weak and easily disturbed by external factors such as
electro-magnetic interference (EMI) from chargers, PWM signals from power
convertors, motors, LEDs etc, temperature changes or just placing a large
conductive object near the sensor. FIXME: add references to conductive noise
from chargers?

In addition to that, many people in the maker and DIY community struggle with
these sensors due to a lack of a library that offers advanced signal processing
for capacitive sensor signals as well as not sufficient insight in the
theoretical background on electrical field propagation and potential
disturbances.

This article describes the design of an open-source library for
Arduino-compatible boards that allows novice users to get started with
capacitive sensing as well as advanced researchers to experiment with different
measurement and signal processing techniques.

\section{Overview of existing open-source libraries}
There are two popular methods to measure self capacitance: the first is use an
RC or LC resonator and measure the time it takes to charge or discharge the
sensor (oscillator based), the second is to inject a known amount of charge
(for example by using a reference capacitor) and measuring the resulting
voltage on the sensor (charge transfer based).

Once of the most popular libraries for capacitive sensing today is the
CapacitiveSensor library from Paul Stoffregen \cite{CapacitiveSensorWebsite}.
This library uses a large resistor (typically a few $\textrm{M}\Omega$) to
create an RC resonator and measures the time it takes to charge and discharge
the sensor. Similarly, the FastTouch library from Adrian Freed
\cite{FastTouchWebsite} uses the internal pullup resistor on GPIO pins. This
has the advantage that it does not require an extra GPIO pin and resistor at
the disadvantage of a much lower signal to noise ratio.

In contrast, Jens Geisler \cite{QTouchADCArduinoWebsite}, Martin Pittermann
\cite{ADCTouchWebsite} and Nico Hood \cite{AnalogTouch} all use the charge
transfer method where they use the internal sample and hold capacitor from the
internal ADC as reference capacitor. This method is extensively documented as
CVD method by Microchip in FIXME: add reference to application notes.

What all of these libraries have in common though, is that they have little to
no signal processing to improve the robustness of the capacitive sensing for
external influences. Additionally, these libraries are typically targeted to
only one or two hardware platforms (usually Arduino UNO and one or two other
boards).

TouchLib was therefore designed from the ground up for robust capacitive
sensing by using different signal processing methods on different levels of
signal interpretation. Additionally, the core capacitive sensing function was
designed to easily add other methods for capacitive sensing or to extend the
current sensing functions. This feature makes the library ideal for research on
capacitive sensing methods and also makes it easy to port TouchLib to new
platforms.

\section{TouchLib design}
asdaf

\section{FOO}
- what are capacitive sensors?
- why touchlib? (existing open source libraries are not portable and lack signal processing for robust sensing + needed framework for research)
- design goals
- signal processing (pseudo-differential sampling, pseudo-random scanning order, integration, low pass filter, button state machine)
- code generator
- resistive sensing (PrePre, Z-touch)
- user experiences (e-Textile summercamp 2017, e-Textile springbreak 2018, WeMake makerspace Milano, student groups at TU/e)
- products (TrueBlue Box, Beleef kussen)
- future work (documentation on theoretical background of capacitive sensing, documentation of library, better noise measurement, true hierarchical state machine, slider / wheel / trackpad, pattern generator, add support for capacitive sensing using a resistor, improvements to current capacitive sensing methods, reduce memory consumption)
\end{document}
