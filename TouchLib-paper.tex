\documentclass[format=sigchi, screen=true, review=true]{acmart}
% rubber: clean TouchLib-paper.brf TouchLib-paper.err TouchLib-paper.out TouchLib-paper.thm TouchLib-paper-blx.bib TouchLib-paper.run.xml TouchLib-paper.xcp
% rubber: depend TouchLib-paper.bib
% rubber: module bibtex
% rubber: module index
\usepackage[english]{babel}
%\usepackage[square,numbers]{natbib}
\usepackage{natbib}

\begin{document}
\bibliography{TouchLib-paper}
\bibliographystyle{abbrvnat}
\acmConference[CHI]{ACM CHI Conference on Human Factors in Computing Systems}{May 04--09, 2019}{Glasgow, UK}
\title{TouchLib: a portable and user-friendly library for robust capacitive sensing}
\author{Admar Schoonen}
\affiliation{Eindhoven University of Technology, department of Industrial Design}
\email{admar@familieschoonen.nl}
\maketitle
\section{Introduction}
Capacitive sensors as part of a human interface device (HID) have been used for
a long time. Their attributes such as low power, low cost, ability to place the
sensor behind smooth and flush surfaces and the fact that no force at all is
required to activate the sensor, giving the user an almost magical experience,
all contribute to their widespread use. FIXME: add references to early
applications, iPod and iPhone?

However, capacitive sensors are also known to be difficult to work with since
the signals can be very weak and easily disturbed by external factors such as
electro-magnetic interference (EMI) from chargers, PWM signals from power
convertors, motors, LEDs etc, temperature changes or just placing a large
conductive object near the sensor. FIXME: add references to conductive noise
from chargers?

In addition to that, many people in the maker, DIY and artist community struggle with
these sensors due to a lack of a library that offers advanced signal processing
for capacitive sensor signals as well as not sufficient insight in the
theoretical background on electrical field propagation and potential
disturbances.

This article describes the design of an open-source library for
Arduino-compatible boards that allows novice users to get started with
capacitive sensing as well as advanced researchers to experiment with different
measurement and signal processing techniques.

\section{Overview of existing open-source libraries}
There are two popular methods to measure self capacitance: the first is use an
RC or LC resonator and measure the time it takes to charge or discharge the
sensor (oscillator based), the second is to inject a known amount of charge
(for example by using a reference capacitor) and measuring the resulting
voltage on the sensor (charge transfer based).

Once of the most popular libraries for capacitive sensing today is the
CapacitiveSensor library from Paul Stoffregen \cite{CapacitiveSensorWebsite}.
This library uses a large resistor (typically a few $\textrm{M}\Omega$) to
create an RC resonator and measures the time it takes to charge and discharge
the sensor. Similarly, the FastTouch library from Adrian Freed
\cite{FastTouchWebsite} uses the internal pullup resistor on GPIO pins. This
has the advantage that it does not require an extra GPIO pin and resistor at
the disadvantage of a much lower signal to noise ratio.

In contrast, Jens Geisler \cite{QTouchADCArduinoWebsite}, Martin Pittermann
\cite{ADCTouchWebsite} and Nico Hood \cite{AnalogTouch} all use the charge
transfer method where they use the internal sample and hold capacitor from the
internal ADC as reference capacitor. The CVD method has as advantage that it
does not require any external components and results in high signal to noise
ratios but it only works on analog input pins. This method is extensively
documented as CVD method by Microchip in FIXME: add reference to application
notes.

FIXME: add Z-patch

What all of these libraries have in common though, is that they have little to
no signal processing to improve the robustness of the capacitive sensing for
external influences. Additionally, these libraries are typically targeted to
only one or two hardware platforms (usually Arduino UNO and one or two other
boards).

FIXME: no easy method for tuning capacitive and resistive sensors

TouchLib was therefore designed from the ground up for robust capacitive
sensing by using different signal processing methods on several levels of
signal interpretation. Additionally, the core capacitive sensing function was
designed to easily add other methods for capacitive sensing or to extend the
current sensing functions. This feature makes TouchLib both ideal for research
on capacitive sensing methods as well as easy to port to other platforms.

\section{TouchLib design process}
how this design process led to easy to use multi-platform library:
\begin{itemize}
	\item iterative design method with experience from students and artists from the design, and research community at TU/e, e-textile summercamps, milan workshop as well as experts outside these fields (philip ross got a private milan workshop from troy, paul van tilburg, abbe de groot)
	\item reflective design
	\item code generator would not have been designed
	\item resistive sensing would not have been added if it was not a research library
	\item ports to different platforms would not have been added if resistive sensing was not already there and if users from our community at TU/e would not have asked for it
\end{itemize}

how the design process influenced the design specifically:
\begin{itemize}
	\item workshop on resistive sensing in 2015, realisation that there is a need for robust capacitive sensing library
	\item research on combinig capacitive and resistive sensing led to adding resistive sensing as an alternative acquisition method in 2016
	\item first workshop with students at tue in 2016, led to the code generator
	\item workshop at e-textile summercamp 2017 and questions from students led to port to teensy and particle photon
	\item workshop at e-textile spring break 2018 led to port to esp32
\end{itemize}

\section{TouchLib design}
\begin{itemize}
\item drawing of capacitive touch signal
\item necessity of baseline calibration at startup
\item signal processing:
	\begin{itemize}
	\item pseudo-differential sampling
	\item pseudo-random scanning order
	\item integration
	\item low pass filter
	\item button state machine
	\end{itemize}
\item selectable integration filters
\item state machine
\item sensing methods (CVD, touchRead, analogRead)
\item ports:
	\begin{itemize}
	\item Arduino 8-bit AVR, including UNO, Mega, Lilypad, FLORA and many others (ATmega328p, ATmega2560, ATmega32u4)
	\item Teensy 3.x (Freescale MK-series)
	\item Particle Photon series (STM32F2xx)
	\item ESP32DEV (ESP32)
	\end{itemize}
\end{itemize}

\section{Combining capacitive sensing with resistive sensing}
Adrian Freeds work, PrePre (e-textile summercamp 2016), Z-patch

\section{Code generator}
\begin{itemize}
\item no specific program needed on host PC; code generator runs fully on target platform:
	\begin{itemize}
	\item no need to install additional software
	\item works on every computer that can run Arduino IDE
	\item user does not need to select a specific target platform in the code generator platform as this is known at run-time
	\end{itemize}
\item user ends up with ready to use code for their next capacitive touch project
\item visualization of touch performance after tuning
\end{itemize}

\section{Applications}
\begin{itemize}
\item user experiences
	\begin{itemize}
	\item e-Textile summercamp 2016 (no touchlib, just pre-pre)
	\item e-Textile summercamp 2017
	\item e-Textile springbreak 2018
	\item WeMake makerspace Milano
	\item student groups at TU/e
	\item Expert interviews with Hannah, Lara, Liza, Beam, Becky, Zihao Yang, Philip Ross, Abbe, Serge
	\item 3D printed and textile projects from Troy (note different materials: embroidery, conductive TPU, e-textiles)
	\item allows faster design - build - evaluation cycles 
\end{itemize}
\item commercial products
	\begin{itemize}
	\item TrueBlue Box
	\item Beleef kussen
	\end{itemize}
\end{itemize}

\section{Future work}
In no particular order:
\begin{itemize}
\item documentation on theoretical background of capacitive sensing
\item documentation of API
\item better noise measurement
\item true hierarchical state machine
\item slider / wheel / trackpad
\item pattern generator
\item add support for capacitive sensing using a resistor
\item objective comparison between performance (SNR, dynamic range) different sensing methods and platforms
\item improvements to current capacitive sensing methods
\item reduce memory consumption
\item automated testing
\end{itemize}

\section{FOO}

\begin{itemize}
\item what are capacitive sensors?
\item why touchlib? (existing open source libraries are not portable and lack signal processing for robust sensing + needed framework for research)
\item design goals
\item code generator
\item resistive sensing (PrePre, Z-touch)
\item user experiences (e-Textile summercamp 2017, e-Textile springbreak 2018, WeMake makerspace Milano, student groups at TU/e)
\item products (TrueBlue Box, Beleef kussen)
\item future work (documentation on theoretical background of capacitive sensing, documentation of library, better noise measurement, true hierarchical state machine, slider / wheel / trackpad, pattern generator, add support for capacitive sensing using a resistor, improvements to current capacitive sensing methods, reduce memory consumption, automated testing)
\end{itemize}
\end{document}
